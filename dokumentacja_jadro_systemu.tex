%TODO:
%	-
%	-
\documentclass[a4paper,12pt]{article}


% to jest do polskich znakow
\usepackage[polish]{babel}
\usepackage[T1]{polski}
\usepackage[utf8]{inputenc} 
\usepackage{indentfirst}%wcięcie pierwszego wiersza
\usepackage{listings}%listing kodu
        
\textheight=24.5cm
\textwidth=17cm
\topskip=5mm
\topmargin=-15mm
\leftmargin=-1mm
\oddsidemargin=-1mm      %10mm
\evensidemargin=-1mm     %10mm
\renewcommand{\baselinestretch}{1}


\begin{document}
\thispagestyle{empty}

\noindent
\unitlength=1mm
\fboxrule=1mm
\setlength{\fboxrule}{3pt}\fbox{\LARGE\sf WT/NP/17.05}\\[+2mm]
\begin{tabular}{ll}
\begin{minipage}[t]{110mm}
{\LARGE\sf 
Jakub Pomykała 209897}
\end{minipage}
 &
\begin{minipage}[t]{55mm}
{\Large\sf 
\begin{tabular}{ll}
Ocena:  & \raisebox{-3mm}{\framebox(30,9)[cc]{}}\\
\end{tabular}
}
\end{minipage}
\end{tabular}

\hfill {\Large\sf Oddano: \raisebox{-3mm}{\framebox(30,9)[cc]{}}}\\[+10mm]

\begin{center}
{\huge\sf Proste jądro systemu operacyjnego}\\[+10mm]
{\Large\sc 
Architektura Komputerów 2 -- projekt\\
%Architektura komputer\'ow (2) -- projekt\\
INF 2014/15\\[+10mm]}
\end{center}

\noindent
{\sc 
\hspace*{70mm}Prowadz\k{a}cy:\\
\hspace*{70mm}dr inż. Tadeusz Tomczak\\
}

\newpage
\tableofcontents % to jest do spisu treści
\newpage
	
	\section{Wprowadzenie}
		\subsection{Plan projektu i osiągnięcia}
		Projekt polegał na napisaniu prostego jądra systemu operacyjnego, przejścia w tryb chroniony i przełączaniu zadań za pomocą przerwań wywoływanych poprzez zegar systemowy. Kod źródłowy jądra został napisany w Turbo Assemblerze i uruchamiany jest w DOSBoxie 0.74. Początkowy plan zakładł napisanie jądra, bootloadera i uruchamianie jądra na komputerze PC z procesorem Intel Pentium z dyskietki. Niestety nie udało mi się skończyć pisać bootloadera, dlatego jądro uruchamiane jest w emulatorze DOSBox. Plan prac wyglądał następująco:
			\begin{itemize}
				\item{przygotowanie środowiska pracy oraz narzędzi}
				\item{przełączenie procesora w tryb chroniony}
				\item{obsługa pamięci rozszerzonej}
				\item{obsługa przerwań i wyjątków}
				\item{przełączanie zadań przez przerwania czasomierza systemowego}	
			\end{itemize}

	\subsection{Podstawowe pojęcia}
\begin{enumerate}
				\item{\textbf{tryb rzeczywisty} - jest to tryb pracy mikroprocesorów z rodziny procesorów x86, w którym procesor pracuje jak Intel 8086. Tryb ten nie zapewnia ochorny pamięci przed użyciem jej przez inny proces oraz obsługi wielozadaniowości. Dostępna jest jedynie 1-megabajtowa przestrzeń adresowa}

				\item{\textbf{tryb chroniony procesora} - tryb pracy procesora, który umożliwia adresowanie pamieci przekraczającej 1-megabajt pamięci, sprzętowa ochrona pamięci, wsparcie w przełączeniu kontekstu procesora, stronnicowanie pamięci (32 bitowe procesory) }

				\item{\textbf{deskryptor} - 64-bitowa struktura danych w której przechowywane są informacje na temat miejsca w pamięci danego segmentu, typu, rozmiaru, zasady dostępu do segmentu oraz pozostałe informacje przydatne przy dostępie do segmentu w trybie chronionym procesora. }

				\item{\textbf{tablice deskryptorów} - w trybie chronionym posługujemy się tablicami deskryptorów, wyróżniamy trzy podstawowe struktury: }
\begin{itemize}
						\item{Global Descriptor Table (GDT) - globalna tablica, zawiera deskryptory, które mogą być wykorzystane przez dowolne zadanie w systemie. Przechowują pamięć ekranu oraz ogólnie dostępne segmenty kodu i danych}
						\item{Local Descriptor Table (LDT) - lokalna tablica, zawiera deskryptory dostępne tylko dla konkretnego zadania}
						\item{Interrupt Descriptor Table (IDT) - tablica deskryptorów przerwań, użwana do poprawnego reagowania na przerwania oraz wyjątki}
					\end{itemize}

\item{\textbf{rejestry segmentowe} - zawierają adresy bazowe tablic systemowych, służą do organizacji segmentacji w trybie chronionym}
\begin{itemize}
\item{Global Descriptor Table Registers (GDTR) - liniowy adres bazowy i rozmiar globalnej tablicy deskryptorów}
\item{Interrupt Descriptor Table Registers (IDTR) - liniowy adres bazowy i rozmiar tablicy deskryptorów przerwań}
\item{Local Descriptor Table Registers (LDTR) - selektor segmentu tablicy deskryptorów lokalnych}
\item{Task Registers (TR) - rejestr stanu zadania, selektor stanu zadania}
\end{itemize}
				\item{\textbf{selektor} - w trybie chronionym procesora selektory są umieszczone w rejestrach segmentowych. Format selektora prezentuje się następująco:}

\begin{itemize}
\item{INDEX - indeksu w tablicy deskryptorów, bity numer 15 - 3}
\item{TI - wyróżnika tablicy, czy tablica jest globalna (0) czy lokalna (1), bit numer 2}
\item{RPL - poziomu uprzywilejowania, bity numer 1 - 0}
\end{itemize}


\item{\textbf{segmentacja pamięci w trybie chronionym} - każdy segment danych bądź stosu jest opisany parametrami:
\begin{itemize}
\item{lokalizacja w przestrzeni adresowej pamięci}
\item{zasady dostępu}
\item{8 bajtowa struktura danych nazywana deskryptorem}
\end{itemize}
Tablice mogą zawierać od 8 bajtów do 64kB (8192 deskryptory)

Odwołanie do odpowiedniego deskryptora wykonuje się za pomocą selektora zapisanego w jednym z 16 bitowych rejestrów segmentowych: 

\begin{itemize}
\item{rejestr DS, ES, FS, GS - segement musi mieć zezwolenie tylko do odczytu}
\item{rejestr SS - musi mieć ustawione prawa zapisu oraz odczytu}
\item{rejestr CS - wymaga prawa kodu wykonywalnego}
\end{itemize}
FS oraz GS są dostępne tylko w trybie chronionym. W przypadku wpisania błędnego selektora do rejestru segmentowego otrzymamy błąd ''Ogólnego naruszenia ochrony''. }
	
\item{\textbf{rozmieszczenie segmentów w pamięci fizycznej} - wyznaczanie adresu fizycznego na podstawie adresu logicznego wygląda następująco, np. ABCDh:1234h odpowiada następujący adres fizyczny: ABCD0h + 1234h = ACF04h = 708356(10). Adresy segmentów mogą się częściowo nakładać, a nawet w pełni pokrywać ze względu na 16 bitowy offset. }


\item{\textbf{Linia A20 (bramka A20)} - w trybie rzeczywistym było nie wiecej niż 20 fizycznych linii adresowych, w celu zachowania kompatybilności jest ona domyślnie nieaktywna}

	\item{\textbf{przerwanie} - jest to sygnał, który powoduje zmianę przepływu sterowania, niezależnie od aktualnie wykonywanego programu. W przypadku pojawienie się przerwania wstrzymywany jest aktualne wykonywane zadanie i następuje skok do innego miejsca w kodzie, np. procedury. Procedura ta wykonuje czynności związane z obsługą przerwania i na końcu wydaje instrukcję powrotu z przerwania, która powoduje powrót do programu realizowanego przed przerwaniem.
Rozróżniamy kilka typów przerwań:}
\begin{itemize}
\item{programowe - wywoływane przez programistę, instrukcją INT + kod przerwania, lub w przypadku operacji niedozwolonych, np. dzielenie przez zero}
\item{sprzętowe - generowane przez urządzenia zewnętrzne, np. obsługa klawiatury, czyli wciśniecie jakiegoś klawisza, może to też być drukarka, myszka, dysk twardy itp.}
\item{wyjątki - generowane przez zewnętrzne układy procesora}
\end{itemize}

\item{\textbf{kontroler przerwań} - układ obsługi przerwań w komputerach PC jest zbudowany z dwóch połączonych kaskadowo układów 8259A, dzięki temu możliwa jest obsługa 15 przerwań sprzętowych - wejście IRQ2 układu master jest połączone z wyjściem układu slave. Kontroler klawiatury znajduje się na linii IRQ1, a czasomierz systemowy na linii IRQ0}
				\item{\textbf{czasomierz systemowy (lub zegar systemowy)} - jest to fizyczne urządzenie znajdujące się na płycie głównej komputera, odpowiedzialne za dostarczanie aktualnego czasu i daty do komputera. Odpowiada również za dostarczanie sygnałów synchronizujących działanie podzespołów komputera z dokładnością do tysięcznych części sekundy.}

\item{\textbf{zadanie (ang. task)} - rozumiemy jako wykonywany program lub niezależny jego fragment }

\item{\textbf{Task State Segment (TSS) } - segment stanu zadania jest rekordem wchodzącym w skład segmentu danych lub oddzielonym segmentem o niewielkim rozmiarze. Każde zadanie ma swój segment stanu zadania. Segmentowi TSS odpowiada systemowy deskryptor tego segmentu, przechowywany w globalnej tablicy deskryptorów. Struktura jest analogiczna do deskryptora pamięci, jedyna różnica polega na różnych kodach typów segmentów. }
			
\end{enumerate}




	\subsection{Środowisko pracy i narzędzia}
	Jądro systemu było testowane za pomocą programu DOSBox 0.74 na komputerze z systemem Windows 8.1 x64. Program DOSBox 0.74 jest pełnym emulatrem procesora Intel 80386 udostępnianym na licencji GNU GPL. Kod jądra był asemblowany za pomocą TASM.exe (Turbo Assembler) oraz linkowany za pomocą TLINK.exe (Turbo Linker). W celu uruchomienia projektu należy zasemblować plik MAIN.ASM poleceniem TASM MAIN.ASM, zlinkować TLINK MAIN.OBJ i uruchomić MAIN.EXE w emulatorze DOSBox. 

	\section{Praca jądra systemu w trybie chronionym}


	\subsection{Przełączanie procesora w tryb chroniony}
Procesor na początku swojego działania znajduje się w trybie rzeczywistym, żeby przełączyć go w tryb chroniony musimy zdefiniować strukturę globalnej tablicy deskryptorów (GDT).

\begin{itemize}
\item{GDT\_NULL - wymagany do poprawnego obliczenia całej zajmowanej pamięci przez deskryptory}
\item{GDT\_DANE - opisuje segment danych, możliwy odczyt i zapis danych (flaga 92h)}
\item{GDT\_PROGRAM - opisuje segment programu, kod z tego deskryptora moze być jedynie wyknywany (flaga 98h)}
\item{GDT\_STOS - segment stosuu z flagą 92h o rozmiarze 256bajtów}
\item{GDT\_EKRAN - segment karty graficznej, rozdzielczość 25 wierszy i 80 kolumn, rozmiar segmentu 4096 bajtów i adres bazowy równy B800h}
\item{GDT\_TSS\_0, GDT\_TSS\_1, GDT\_TSS\_2 -  deskryptory zadań wykorzystywanych w jądrze}
\item{GDT\_MEM - deskryptor o rozmiarze 64kB, umieszczony pod adresem 40000h (4MB)}
\item{GDT\_SIZE - wymagany do poprawnego obliczenia całej zajmowanej pamięci przez deskryptory}

\end{itemize}

Następnie w rejestrze CR0 należy ustawić pierwszy bit (tzw. bit PE - Protection Enable) na 1. Można to zrobić za pomocą instrukcji SMSW lub MOV. Od tej pory nasz procesor pracuje w trybie chronionym. Żeby powrócić do trybu rzeczywistego wystarczy, że wyzerujemy bit PE w rejestrze CR0.

\subsection{Pamięć rozszerzona}
W celu zaadresowania segmentu pamieci pod adresem większym niż 1MB musimy aktywować linię A20, wiele współczesnych BIOS-ów potrafi to zrobić za pomocą odpowiedniego przerwania. Funkcja 24h przerwania 15h, w zależności od wartości przekazanej w rejestrze AL, może wykonywać następujące czynnośc: 
\begin{itemize}
\item{AL = 0 - deaktywacja linii A20}
\item{AL = 1 - aktywacja linii A20}
\item{AL = 2 - zwrócenie informacji o stanie linii A20}
\item{AL = 3 - zwrócenie informacji o możliwości aktywacji linii A20 przez port 92h}
\end{itemize}

Dokładny kod aktywacji linii A20 został przedstawiony w pliku PODST.TXT (linie 33-98)


\subsection{Obsługa przerwań i wyjątków}
Aby wprowadzić obsługi przerwań musimy:
\begin{itemize}
\item{utworzyć tablicę deskryptorów przerwań IDTR - MAIN.ASM (linie 17 - 23)}
\item{umieścić w niej adresy procedur obsługi wyjątków - PODST.TXT (linie 247 - 255) }
\item{załadować adres tablicy IDT do IDTR - PODST.TXT (linie 258 - 274)}
\item{odpowiednie skonfigurowanie kontrolera przerwań - PODST.TXT (linie 228 - 245)}
\end{itemize}

		\subsection{Oprogramowanie kontrolera przerwań}
Zaprogramowanie pracy kontrolera przerwań polega na zamaskowaniu nieobsługiwanych programowo przerwań sprzętowych (np. myszka czy dysk twardy), zależy nam jedynie na obsłudze zegara systemowego. W pliku PODST.TXT makropolecenie KONTROLER\_PRZERWAN, które przyjmuje jako parametr maskę przerwań układu. Wartość 1 na danej pozycji oznacza zablokowanie przerwań na tej linii. Czasomierz systemowy, który posłuży nam do wywoływania przerwań systemowych znajduje się na linii IRQ0. W takim razie użyjemy maski FEh, która binarnie wynosi 1111 1110. Co oznacza że jedynymi przerwaniami sprzętowymi jakie będziemy otrzymywać będą przerwania z czasomierza systemowego.

\subsection{Obsługa przerwania pochodzącego z czasomierza systemowego}
W momencie poprawnej konfiguracji kontrolera przerwań ostatnim krokiem do obsługi przerwań jest odblokowanie ich otrzymywania za pomocą instrukcji STI. W tym momencie z każdym przerwaniem czasomierza program będzie przenosić się do linii 60 w pliku MAIN.ASM, gdzie następuje obsługa przerwania.
Obsługa przerwania to przełączenie zadania na jedno z dwóch za pomocą intrukcji porównania CMP i skoku warunkowego JE do opowiedniej etykiety. 

\begin{lstlisting}
	CMP   AKTYWNE_ZADANIE,1
	JE    ETYKIETA_ZADANIE_1  

	;...
		
	ETYKIETA_ZADANIE_1:
	MOV AKTYWNE_ZADANIE, 0	
	JMP DWORD PTR T0_ADDR
\end{lstlisting}

W momencie skoku do odpowiedniego zadania w pliku MAIN.ASM
\begin{itemize}
\item{ZADANIE\_1 - linie 148 - 163}
\item{ZADANIE\_2 - linie 166 - 182}
\end{itemize}
W obu zadaniach wywoływane jest przerwanie programowe (instrukcja INT) którego obsługa polega na wyświetleniu informacji o aktywnym zadaniu. 
Następnie wykonywane jest makro OPOZNIENIE z pliku PODST.TXT  (linie 312 - 326) przy pomocy dwóch zagnieżdzonych pętli. Sygnał zakończenia przerwania
\begin{lstlisting}
	MOV   AL, 20H
	OUT   20H, AL
\end{lstlisting}

czyli informacja dla kontrolera przerwań o zakończeniu obsługi przerwania poprzez zapis wartości 20H na port 20H. Skok na początek aktualnie wykonywanego zadania
\begin{lstlisting}
	JMP ZADANIE_2_PETLA
\end{lstlisting}
W momencie przyjscia kolejnego przerwania jądro znów znajdzie się na linii 60 i całą procedura rozpocznie się od nowa. 

\subsection{Przełączenie zadań z wykorzystaniem przerwań czasomierza systemowego}

Podczas przełączania zadania procesor, następuje zmiana selektora w rejestrze segmentowym CS, zapamiętywany jest kontekst bieżącego zadania, a następnie odczytywany jest TSS kontekst nowego zadania, zawierający selektor segmentu i offset, od którego należy rozpocząć jego realizację. Kolejny krok to rozkaz skoku odległego. W jądrze użyto jedynie globalnych deskryptorów, dlatego nie było problemu ze zmianamy poziomu uprzywilejowania deskryptorów. 

	\section{Zakończenie}
\subsection{Wnioski i możliwości dalszego rozwoju jądra}
Realizacja projektu pozwoliła mi na dokładniejsze poznanie procesrów jakie zachodzą we współczesnych systemach operacyjnych. Dzięki praktyce lepiej poznałem teorię architektury komputerów, mogłem dowiedzieć się jak działa jeden z najważniejszych elementów komputera, czyli procesor. Dzięki podziałowi projektu na kilka plików tekstowych z kodem źródłowym, projekt jest skalowalny. Z łatwością można dodać do niego obsługę klawiatury, czy innych urządzeń zewnętrznych.

\section{Listing kodów źródłowych}

\lstset{
	numbers=left, 
	language={[x86masm]Assembler}
}

\subsection{\textit{DEFSTR.TXT} - struktura deskryptorów}
\lstinputlisting{DEFSTR.TXT}

\subsection{\textit{MAIN.ASM} - kod główny}
\lstinputlisting{MAIN.ASM}

\subsection{\textit{PODST.TXT} - podstawowe funkcje jądra}
\lstinputlisting{PODST.TXT}

\subsection{\textit{OBSLPUL.TXT} - kod obsługi pułapek}
\lstinputlisting{OBSLPUL.TXT}

\subsection{\textit{TXTPUL.TXT} - tekst i atrybuty użyte w kodzie obsługi pułapek}
\lstinputlisting{TXTPUL.TXT}

\subsection{\textit{RODZPUL.TXT} - lista pułapek}
\lstinputlisting{RODZPUL.TXT}


	
\section{Bibiliografia}
\begin{thebibliography}{999}
\bibitem{aa1} W. Stanisławski, D.Raczyński 
{\em Programowanie systemowe mikroprocesorów rodziny x86},
PWN, Warszawa 2010. ISBN 978-83-01-16383-9.

\bibitem{aa2} J. Biernat,
{\em Architektura komputerów}, 
Oficyna Wydawnicza Politechniki Wrocławskiej, Wrocław 2005. ISBN 83-7085-878-3.

\bibitem{aa3} G.Syck,
{\em Turbo Assembler - Biblia użytkownika}, 
LT\&P, Warszawa 1994. ISBN 83-901237-2-X.

\bibitem{aa4} J. Bielecki,
{\em Turbo Assembler }, 
PLJ, Warszawa 1991. ISBN 83-85190-10-4.

\end{thebibliography}	


\end{document}

